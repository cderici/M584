\documentclass{article}

\usepackage[utf8]{inputenc}
\usepackage{color}
\usepackage[left=1in,right=1in,top=1in,bottom=1in]{geometry}
\usepackage{amsmath, amssymb}
%\usepackage{mathabx}
\usepackage{fancyhdr}
\pagestyle{fancy}
\fancyhf{}
\fancyhead[R]{Ambrose Bonnaire-Sergeant \& Caner Derici}
\fancyfoot[C]{\thepage}

% \usepackage{lipsum}

\newcommand{\HRule}{\rule{\linewidth}{0.5mm}}
\newcommand{\Hrule}{\rule{\linewidth}{0.3mm}}


% couldn't find the symbol,
% this is a copy/paste from a stackexchange answer
\providecommand{\dotdiv}{% Don't redefine it if available
  \mathbin{% We want a binary operation
    \vphantom{+}% The same height as a plus or minus
    \text{% Change size in sub/superscripts
      \mathsurround=0pt % To be on the safe side
      \ooalign{% Superimpose the two symbols
        \noalign{\kern-.35ex}% but the dot is raised a bit
        \hidewidth$\smash{\cdot}$\hidewidth\cr % Dot
        \noalign{\kern.35ex}% Backup for vertical alignment
        $-$\cr % Minus
      }%
    }%
  }%
}

\makeatletter% since there's an at-sign (@) in the command name
\renewcommand{\@maketitle}{
  \parindent=0pt% don't indent paragraphs in the title block
  
  {\Large \bf \@title}
  
  \Hrule%
    
  \textit{\@author \hfill \@date}
  \par
}
\makeatother% resets the meaning of the at-sign (@)

\title{M584 - Problem Set 2}
\author{}
\date{Ambrose Bonnaire-Sergeant \& Caner Derici}

\begin{document}
%\pagenumbering{gobble}

\maketitle% prints the title block

\paragraph{1) x1B.16} For \textit{restriction}, let $F(u,n+1)$ be the function that constructs $u$ up to the $n^{th}$ factor.

\[
\begin{array}{rcl}
F(u,0) & = & \langle\rangle \\

F(u, n+1) & = & append(F(u,n),(u)_n)
\end{array}
\]

So $F(u,0) = \langle\rangle$, $F(u,1) = \langle u_0 \rangle$, $F(u,2) = \langle u_0, u_1 \rangle$, and so on. Then $u \upharpoonright i$ is defined as follows.

\[
u \upharpoonright i = \left\{\begin{array}{lr}
    F(u,i), & \text{if } i\leq n \\
    0, & \text{otherwise}
  \end{array}\right\}
\]

We could also define it just as a division as follows.

\[
u \upharpoonright i = \left\{\begin{array}{lr}

  quot(u, \Pi_{i\leq j\leq lh(u)} p_j^{(u)_j+1}), & \text{if } u=\langle u_0, \ldots, u_{n-1}\rangle with i\leq lh(u) \\

    0, & \text{otherwise}
  \end{array}\right\}
\]

\paragraph{} For \textit{concatenation}, let $F(u,v)$ be the function that \emph{appends} $v$ to $u$ one by one.

%% I don't like this (my solution was the multiplication one below), if you have a better solution, just write that version
\[
\begin{array}{rcl}
F(u,\langle\rangle) & = & u \\

F(u,v) & = & append(F(u, v \upharpoonright lh(v)\dotdiv 1), (v)_{lh(v)\dotdiv 1})
\end{array}
\]

So e.g. \[
\begin{array}{rcl}
F(\langle u_0 \rangle,\langle v_0, v_1 \rangle) & = & append(F(\langle u_0 \rangle, \langle v_0 \rangle), v_1) \\

& = & append(append(F(\langle u_0 \rangle,\langle\rangle), v_0), v_1) \\

& = & append(append(\langle u_0 \rangle, v_0), v_1) \\

& = & append(\langle u_0, v_0 \rangle, v_1) \\

& = & \langle u_0,v_0,v_1 \rangle \\

\end{array}
\]

Then define $u * v$ as follows.

\[
u * v = \left\{\begin{array}{lr}
    F(u,v), & \text{if } u = \langle u_0, \ldots, u_{n-1} \rangle, v = \langle v_0, \ldots, v_{m-1} \rangle \\
    0, & \text{otherwise}
  \end{array}\right\}
\]

Similarly to the \textit{restriction}, we could define the \textit{concatenation} also as a multiplication as follows.

\begin{equation*}
u * v = u.\Pi_{0\leq j < lh(v)} p_{n+j}^{(v)_j+1}
\end{equation*}
  
\paragraph{2) x1B.20}

\paragraph{} We know that

\begin{equation*}
  X_P(y,\overrightarrow{x}) = X_H(\langle X_P(0,\overrightarrow{x}), \ldots, X_P(y-1,\overrightarrow{x}) \rangle, \overrightarrow{x})
\end{equation*}

So in order to show that $X_P$ is primitive recursive, instead of trying to define $X_H$ by primitive recursion in terms of $X_P$ (for which we try to define $X_H$ by primitive recursion in the first place), let

\[
\begin{array}{rcl}
  F(0, \overrightarrow{x}) & = & 1 \\
  F(y+1, \overrightarrow{x}) & = & append(F(y,\overrightarrow{x}), X_H(F(y,\overrightarrow{x}), y, \overrightarrow{x}))
\end{array}
\]

and let

\[
\begin{array}{rcl}
  X_P(y, \overrightarrow{x}) & = & (F(y+1,\overrightarrow{x}))_y \\
  & = & \text{ i.e. the last element of } F(y+1,\overrightarrow{x})
\end{array} 
\]



\paragraph{3) x1B.21}


\end{document}
