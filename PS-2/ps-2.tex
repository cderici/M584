\documentclass{article}

\usepackage[utf8]{inputenc}
\usepackage{color}
\usepackage[left=1in,right=1in,top=1in,bottom=1in]{geometry}
\usepackage{amsmath, amssymb}

\usepackage{fancyhdr}
\pagestyle{fancy}
\fancyhf{}
\fancyhead[R]{Ambrose Bonnaire-Sergeant \& Caner Derici}
\fancyfoot[C]{\thepage}

% \usepackage{lipsum}

\newcommand{\HRule}{\rule{\linewidth}{0.5mm}}
\newcommand{\Hrule}{\rule{\linewidth}{0.3mm}}

\makeatletter% since there's an at-sign (@) in the command name
\renewcommand{\@maketitle}{
  \parindent=0pt% don't indent paragraphs in the title block
  
  {\Large \bf \@title}
  
  \Hrule%
    
  \textit{\@author \hfill \@date}
  \par
}
\makeatother% resets the meaning of the at-sign (@)

\title{M584 - Problem Set 2}
\author{}
\date{Ambrose Bonnaire-Sergeant \& Caner Derici}

\begin{document}
%\pagenumbering{gobble}

\maketitle% prints the title block

\paragraph{1) x1B.16}

\paragraph{2) x1B.20}

\newpage

\paragraph{3) x1B.21}
%% \begin{thebibliography}{1}

%% \bibitem{www}
%% E. Başar, C. Derici, Ç. Şenol, \emph{WorldWithWeb: A compiler from world applications to JavaScript}, In proceedings of The Scheme and Functional Programming Workshop\hskip 1em plus 0.5em minus 0.4em\relax Boston, Massachusetts, 2009.

%% \end{thebibliography}


\end{document}
